\chapter{Podsumowanie}

Celem pracy było zaproponowanie autorskiego rozwiązania do detekcji manipulacji zdjęć w postaci modelu sieci głębokiej oraz porównanie go do dwóch kolejnych modeli wykonujących to samo zadanie. Pierwszy z nich miał być oparty o maszynę wektorów nośnych, a drugi miał wykorzystywać uczenie głębokie, a w szczególności ideę jaką jest transfer wiedzy. Wszystkie cele zostały zrealizowane, wyniki prac każdego z modeli zostały sprawdzone dla dwóch zbiorów danych(widoczne na tabli \ref{tab:all_result}). Co więcej zgodnie z tabelami \ref{tab:t_results},\ref{tab:t_results_p} najlepsze wyniki dla każdego ze zbiorów uzyskuje autorskie rozwiązanie(kolejno $\sim91\%$ i $\sim71\%$ dokładności). \\

W dalszej pracy nad zadaniem detekcji manipulacji zdjęć należałoby spróbować zbudować model w oparciu o rozwiązania zaproponowane w artykule \cite{resnet}, w tym między innymi ideę połączeń \textit{skip-connection}. Pozwoliłoby to dodać kolejne warstwy do modelu, co mogłoby pozytywnie wpłynąć na wyniki. Warto by również, spróbować zaproponować rozwiązanie oparte o model sieci GAN \cite{orggan}, szczególnie że można by do takiego rozwiązania skorzystać z modelu generującego sztuczne twarze(opisane w artykule \cite{gan}), co zapewniłoby na dostęp do dużej ilości danych, oraz możliwość sprawdzenia wyników w oparciu o najnowsze i najlepsze aktualnie dostępne rozwiązanie.
